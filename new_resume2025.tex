\documentclass[11pt,a4paper,sans]{moderncv}

% modern themes
\moderncvstyle{classic}                            % Style options: 'casual', 'classic', 'oldstyle', 'banking'
\moderncvcolor{blue}                               % Color options: 'blue' (default), 'orange', 'green', 'red', 'purple', 'grey', 'black'

% character encoding
\usepackage[utf8]{inputenc}                        % Ensure ATS can read it

% adjust the page margins
\usepackage[scale=0.85]{geometry}

% personal data
\name{Deepa}{Tilwani}
\address{Room 524, Artificial Intelligence Institute, AIISC, \\ 5th Floor, 1112 Greene St, Columbia, SC, 29208\\} 
\phone[mobile]{+1 (803) 477-4526}
\email{dtilwani@mailbox.sc.edu}
\social[linkedin]{deepa-tilwani-b758551a0}
\social[googlescholar]{https://tinyurl.com/3b3rerew}

\begin{document}
\makecvtitle


My research focuses on Neurosymbolic AI to develop trustworthy large language models (LLMs) for interdisciplinary and high-stakes applications. I have received accolades for interdisciplinary research in healthcare by developing plausible methods for interpretable and explainable AI. 
%I won the Trainee Best Research Presentation at SCAND for my work on ECG recordings as autism predictors. 
I have accepted papers in top-tier journals and conferences, including AAAI (core A*), IEEE Intelligent Systems (IF: 6.7), and IEEE Journal of Biomedical and Health Informatics (IF: 7.7), where I explore the integration of symbolic reasoning with LLMs. I also developed REASONS, an open-source benchmark dataset for developing attribution and reasoning capabilities in LLMs.  
\section{Education}
\cventry{2022--Present}{Ph.D. in Computer Science and Engineering}{University of South Carolina}{Columbia, SC, USA}{}{}
\cventry{2019--2022}{M.Tech in Computer Science and Engineering}{The LNM Institute of Information Technology}{Jaipur, Rajasthan, India}{}{Thesis: \emph{Predicting Familial Likelihood of Autism Spectrum Disorder in Infancy
Using ECG}}
\cventry{2014--2018}{B.Tech in Computer Science and Engineering}{Govt. Women Engineering College}{Ajmer, Rajasthan, India}{}{}

\section{Skills}
% Dividing into categories for better ATS parsing

\cvitem{Programming Languages}{Python, PyTorch, Keras, TensorFlow, Scikit-learn, NumPy, Pandas, CUDA, GIT, C}
\cvitem{Tools}{Seaborn, Matplotlib, Jupyter, Git, Docker, Matlab, GPU}
\cvitem{Methodologies}{Machine Learning, Deep Learning, Knowledge Graphs, NeuroSymbolic AI, Signal Processing (EEG, fMRI/MRI, ECG), Large Language Models (LLMs), Reterival Augmentation Generation (RAG)}
\cvitem{Soft Skills}{Team Leadership, Project Management, Communication Skills, Cross-Functional}

\section{Work Experience}

% Using 'Work Experience' instead of 'Professional Experience' to ensure ATS parsing
\cventry{2022--Present}{Graduate Research Assistant}{Artificial Intelligence Institute, University of South Carolina}{Columbia, SC, USA}{}{
    \begin{itemize}
        \item Leading research projects focused on LLMs for search and attribution.
        \item Led a project on \emph{ECG Recordings as Predictors of Very Early Autism Likelihood: A Machine Learning Approach} that resulted in \emph{Trainee Best Research Presentation Winner in SCAND Symposium}.
        \item Developed an open source dataset for attribution evaluation \emph{REASONS: A benchmark for REtrieval and Automated citationS Of scieNtific Sentences using Public and Proprietary LLMs.}.
        \item Collaborated with a multidisciplinary team \emph{to develop machine learning benchmarks for neuroimaging data}.
    \end{itemize}
}


\cventry{June 2025--August 2025}{AI Research And Development Intern}{Neural Nest, LLC}
{Remote, USA}
{}{
    \begin{itemize}
        \item Contributed to the development of ARTHEMIS, an AI-driven Virtual Courtroom platform for U.S. arbitration and mediation.  
        \item Developed multi-agent simulation models to enhance tribunal dynamics and decision-making processes.  
        \item Designed workflows that reduced case resolution times from over 60 days to approximately 15 days, improving efficiency by 75%.  
        \item Supported the creation of secure digital courtroom spaces, increasing accessibility for users nationwide.
            \end{itemize}
}


\cventry{2021--2022}{Visiting Research Scholar}{Artificial Intelligence Institute, University of South Carolina}{Columbia, SC, USA}{}{
    \begin{itemize}
        \item Facilitated research within a multidisciplinary neuroscience team by contributing AI expertise, helping to bridge the gap between computational methods and neurocognitive studies.
        \item Provided valuable insights to support the development of AI tools for analyzing neuroimaging datasets.
        \item Gained substantial experience in neuroimaging data (EEG, MRI) processing and analysis, collaborating with neuroscience experts to refine research goals and methodologies.
        \item Studied emerging trends in AI, machine learning, and deep learning, including their applications in neuroscience, leading to enhanced understanding and expertise in both fields.
    \end{itemize}
}
\cventry{2020--2021}{Remote Research Intern}{Artificial Intelligence Institute, University of South Carolina}{Columbia, SC, USA}{}{
    \begin{itemize}
        \item Collaborated on experiments exploring the interaction between ECG and machine learning, assisting in developing new research methodologies.
        \item Assisted in designing experiments that tested AI models' effectiveness in predicting Autism likelihood from ECG.
    \end{itemize}
}

\section{Publications}
% Breaking the section into clear types for ATS readability
\cvitem{Journal Articles}{
    \begin{itemize}
        \item \textbf{Tilwani, D.}, O'Reilly, C., Riccardi, N., Shalin, V.L., den Ouden, D.B., Fridriksson, J., Shinkareva, S.V., Sheth, A.P. and Desai, R.H., (2025). Benchmarking machine learning models in lesion-symptom mapping for predicting language outcomes in stroke survivors. \textbf{Frontiers in Neuroimaging}, 4, p.1573816.
        \item Dalal, S., \textbf{Tilwani, D.}, Gaur, M., Jain, S., Shalin, V., \& Sheth, A. (2024). A Cross Attention Approach to Diagnostic Explainability Using Clinical Practice Guidelines for Depression 2024. \textbf{IEEE Journal of Biomedical and Health Informatics (IF: 7.7)} [\href{https://pubmed.ncbi.nlm.nih.gov/39418143/}{Paper}].
        \item \textbf{Tilwani, D.}, Venkataramanan, R., \& Sheth, A. P. (2024). Neurosymbolic AI Approach to Attribution in Large Language Models. \textbf{IEEE Intelligent Systems 2024 (IF 5.6)} [\href{https://ieeexplore.ieee.org/abstract/document/10777891}{Paper}].
        \item \textbf{Tilwani, D.}, Bradshaw, J., Sheth, A., \& O'Reilly, C. (2023). ECG Recordings as Predictors of Very Early Autism Likelihood: A Machine Learning Approach. \textit{Bioengineering}. [\href{https://pubmed.ncbi.nlm.nih.gov/37508854/}{Paper}]
        \item O'Reilly, C., Oruganti, S. D. R., \textbf{Tilwani, D.}, \& Bradshaw, J. (2023). Model-Driven Analysis of ECG Using Reinforcement Learning. \textit{Bioengineering}.  [\href{https://pubmed.ncbi.nlm.nih.gov/37370627/}{Paper}]
    \end{itemize}
}
\cvitem{Conference Proceedings}{
    \begin{itemize}
         \item \textbf{Tilwani, D.}, O’Reilly, C. (2025). Deep Jansen-Rit Parameter Inference for Model-driven Analysis of Brain Activity. \textbf{Proceedings of Advances in Signal Processing and Artificial Intelligence}, 92.
         \item Mohseni, S.,  Mohammadi, S, \textbf{Tilwani, D.}, Ndawula, G. K., Vema, S., Saxena, Y., Raff, E.,  Gaur, M. Can LLMs Obfuscate Code? A Systematic Analysis of Large Language Models into Assembly Code Obfuscation. \textbf{\textit{Proceedings of AAAI 2025}}. [\href{https://arxiv.org/abs/2412.16135}{Pre-Print}]
        \item Porwal, S., Patel, K. C., \textbf{Tilwani, D.}, \& Bansal, S. K. (2021). A Comparative Study and Tool to Early Predict Diabetes Using Machine and Deep Learning Techniques. \textit{Emerging Trends in Data-Driven Computing and Communications}. [\href{https://link.springer.com/chapter/10.1007/978-981-16-3915-9_29}{Paper}]
    \end{itemize}
}
\cvitem{Tutorials}{
\begin{itemize}
\item Gaur, M., \textbf{Tilwani, D.}, Raff E., Azimi, I., Chadha, A., Neurosymbolic AI for EGI: Explainable, Grounded, and Instructable Generations, In AAAI 2025
\end{itemize}
}
\cvitem{Posters}{
    \begin{itemize}
        \item \textbf{Tilwani, D.}, O'Reilly, C. Exploring Neural Dynamics: A Long Short-Term Memory for Brain Effective Connectivity Analysis in EEG. Discover USC, 2024. [\href{https://github.com/lina-usc/Jansen-Rit-Model-Benchmarking-Deep-Learning/blob/main/Discover_EEG.pdf}{Poster}]
        \item \textbf{Tilwani, D.}, Goswami, R., O'Reilly, C., Riccardi, N., Yang, X., Shalin, V., Shinkareva, S., Sheth, A., \& Desai, H. R. (2023). Predicting Language Outcomes from MRI Post-Stroke: A Machine Learning Approach. \textit{Organization for Human Brain Mapping}, Montreal, Canada. [\href{https://github.com/Deepa-Tilwani/MRI-lesion-sym-mapping/blob/main/poster.pdf}{Poster}]
        \item \textbf{Tilwani, D.}, O'Reilly, C., Bradshaw, J., \& Sheth, A. (2023). Interpretable Machine Learning for Predicting the Likelihood of Autism from Infant ECG Recordings. \textit{SCAND Research Symposium}, Columbia, SC.
        [\href{https://github.com/Deepa-Tilwani/Autism_code/blob/main/SCANDPoster.pdf}{Poster}, Trainee Best Research Presentation Winner]
    \end{itemize}
}
\cvitem{Under Review}{
\begin{itemize}
    \item \textbf{Tilwani, D.}, Saxena, Y., Mohammadi, A., Raff, E., Sheth, A., Parthasarathy, S., \& Gaur, M. (2024). REASONS: A benchmark for REtrieval and Automated citationS Of scieNtific Sentences using Public and Proprietary LLMs. \textbf{(ACL ARR Metareview score 4, Under Review WWW 2025)}
    \item \textbf{Tilwani, D.}, O'Reilly, C., Riccardi, N., Shalin, V., Shinkareva, S., Sheth, A., \& Desai, H. R. (2024). Predicting Language Ability from MRI in Post-Stroke Patients: An Advanced Machine Learning Approach.
    \item \textbf{Tilwani, D.} and O'Reilly, C. Benchmarking Deep Jansen-Rit Parameter Inference: An in Silico Study. arXiv e-prints (2024): arXiv-2406.
   
\end{itemize}
}

\section{Open Source Contributions}
\href{Dataset}{https://zenodo.org/records/18264033}: REASONS: REtrieval and Automated citationS Of scieNtific Sentences. 


\section{Awards and Achievements}
% Used 'and' for better ATS keyword matching
\cvitem{}{
    \begin{itemize}
        \item 2025 Recipient of the 2025 University Travel Award (\$500) for attendance at the International Conference on Advances in Signal Processing and Artificial Intelligence (ASPAI' 2025).
        \item 2025 AAAI-25 Student Scholarship Award (\$1000) 
        \item 2024 EMNLP Diversity and Inclusion Award (\$1000)
        \item 2023 Trainee Best Research Presentation Winner (\$100), SCAND Symposium.
        \item 2023 Research Symposium Third Place Poster Award (\$200), University of South Carolina.
        \item 2021 Nirmala and Jashwantlal Clerk Memorial Scholarship (\$15000), AIISC.
        \item 2020 2nd Prize (\$100), LINZ Ars Festival - BR41N.IO Hackathon.
        \item 2020 2nd Prize (\$300), BR41N.IO: Brain-Computer Interface Designers Hackathon.
        \item 2016 1st Place, Poster Presentation on AR and VR Technology, GWECA.
        \item 2015 3rd Place, Coding Challenge: Toast to Code - C Language, GWECA.
        \item 2012 Silver Prize, National Science Olympiad (NSO).
    \end{itemize}
}

\section{Advising and Mentoring}
% Used 'and' to improve keyword recognition
\cvitem{}{
    \begin{itemize}
        \item Yash Saxena, Galgotias University, Sept 2023- Sept 2024. Project: ``REASON: Reference and Assertions for Consistent Evaluation of Factual/Non-Factual Sentences".
        \item Nethra Gunti, IIIT SriCity, 2022. Project: ``Phase Shift Analysis in Autism Spectrum Disorder: A Video-Based Study of Parent and Object Interactions".
        \item Sai Durga Rithvik Oruganti, University of South Carolina, 2022. Project: ``Phase Shift Analysis in Autism Spectrum Disorder: A Video-Based Study of Parent and Object Interactions".
    \end{itemize}
}
\section{Selected Media Coverage}
% Used 'and' to improve keyword recognition
\cvitem{}{
    \begin{itemize}
        \item \textbf{USC 2024 Newsletter: Pioneering AI to transform autism diagnosis.} \href{https://sc.edu/study/colleges_schools/engineering_and_computing/news_events/news/2024/student_feature_deepa_tilwani.php}{\ul{Link}}
    \end{itemize}
}

\section{Teaching Experience}
% Ensured simple ATS-readable format
\cvitem{}{
    \begin{itemize}
        \item Teaching Assistant, SCINBRE Machine Learning in Python Workshop 2024, University of South Carolina.
        \item Instructor, Introduction to Machine Learning, AIISC High School Summer Camp, 2024.
        \item Instructor, Introduction to Python, AIISC High School Summer Camp, 2023.
        \item Teaching Assistant (2019-2021), The LNM Institute of Information Technology: Computer Networks, Data Structures, DBMS, and Advanced Programming Labs.
    \end{itemize}
}

\section{Community Service}
% Expanded on specific contributions for ATS and clarity

\cvitem{Conference Reviewer}{
    \begin{itemize}
    \item ICWSM 2026
        \item AAAI 20025, 2026
        \item EMNLP 2025
        \item KDD 2025
        \item THE WEB CONFERENCE 2025 (WWW). 
        \item CIKM, KG-STAR Workshop, 2024.
        \item KDD, KIL Workshop, 2024. 
        \end{itemize}}

        
\cvitem{Journal Reviewer}{
    \begin{itemize}
    \item IEEE Internet Computing
         \item Neurosymbolic Artificial Intelligence
         \item ACM Transactions on Computing for Healthcare
        \item ACM Computing Surveys
        \item Scientific Reports
        \item Data Mining and Knowledge Discovery
        \item Frontiers in Neuroimaging
        \item MDPI, Advanced NLP and Machine Translation
    \end{itemize}
}

\cvitem{Program Committee}{
    \begin{itemize}
         \item KG-STAR, CIKM 2024
        \item KIL, KDD 2024
        \item KIL, KDD 2023
    \end{itemize}
}
\cvitem{Voluntary Work}{
    \begin{itemize}
        \item Web and Publicity Chair, KG-STAR Workshop CIKM 2024: Organized events, managed communications, and enhanced visibility of the workshop.
        \item Coordinator, AIISC Retreat, 2023: Organized the institute’s retreat, ensuring participation and facilitating collaborations.
        \item Session Moderator and Publicity Chair, ACM KDD Workshop on Knowledge-infused Learning, 2023: Moderated discussions and Q\&A sessions.
        \item Coordinator, AIISC High School Summer Camp, 2023: Led the planning and execution of the camp, including scheduling and recruitment of instructors.
        \item Student Member, AAAI (2022-Present).
    \end{itemize}
}

\end{document}
